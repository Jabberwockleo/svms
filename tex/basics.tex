\setuplayout[topspace=0.5in, backspace=0.5in, header=24pt, footer=36pt,
  height=middle, width=middle]
\setupfooter[style=\it]
\setupfootertexts[\hfill https://github.com/Jabberwockleo]
\setuppagenumbering[location={header,right}]
\setupbodyfont[11pt]

\starttext

\title{Theoretical Basics}

This chapter gives brief intro to
\startitemize[-]
\item Margin Maximization Method
\item Linear Support Vector Machine
\item Primal/Dual Problems
\item Kernel Method
\item Soft Margin Maximization
\stopitemize

\subject{Margin Maximization Method}

Suppose we have a binary separable data set
\startformula
T = \{(\bold{x_1}, y_1), (\bold{x_2}, y_2), .. (\bold{x_n}, y_n)\}
\stopformula
\startformula
x_i \in \bold{R^n}
\stopformula
\startformula
y_i \in \{+1, -1\}
\stopformula
We are finding a hyperplane $\omega^Tx + b = 0$ that separates points $x_i$ in the feature space.
Points to the side of normal vector pointing to is assigned positive label, vise versa.

We attempt to find the best $\omega$ and $\bold{b}$ (noted as $\bold{\omega^*}$ and $b^*$) that maximize the minimal margin to the hyperplane of all points.

\subject{Geometric Margin}
Recall that we can define a hyperplane (with no bias) in geometric context:

Given a normal vector $\omega$, any point $x$ is one the hyperplane if and only if its corresponding vector $\bold{x}$ is orthogonal to the normal vector. i.e.
\startformula
\omega^T\bold{x} = 0
\stopformula
If we add a bias term $b$, the hyperplane
\startformula
\omega^Tx + b = 0
\stopformula
is explained as:

Given a normal vector $\omega$, any point $x$ is one the hyperplane if and only if the length of the projection of its corresponding vector $\bold{x}$ on the normal vector $\omega$ is $-b'$ (positive if $cos(\theta)$ is positive). i.e.
\startformula
\frac{\omega^T\bold{x}}{||\omega||_2} = -b'
\stopformula
We can scale $b'$ by $||\omega||_2$ and obtain $b$.

The distance of any point $x_i$ to hyperplane $\omega^Tx + b = 0$ is
\startformula
\gamma_i = ||\frac{\omega^Tx_i}{||\omega||_2} + \frac{b}{||\omega||_2}||_2
\stopformula

For any point $x_i$ in T,
\startformula
|| \omega^Tx_i + b || = y_i(\omega^Tx_i + b)
\stopformula

Thus
\startformula
\gamma_i = y_i(\frac{\omega^Tx_i}{||\omega||_2} + \frac{b}{||\omega||_2})
\stopformula

\subject{Primal Problem}
Definition:
\startformula
\max_{\omega, b} \gamma
\stopformula
\startformula
\gamma = \min_{i = 1,...,N}\gamma_i
\stopformula
i.e.
\startformula
\max_{\omega, b} \gamma
\stopformula
\startformula
y_i(\frac{\omega^Tx_i}{||\omega||_2} + \frac{b}{||\omega||_2}) \ge \gamma
\stopformula
Rearranged:
\startformula
\max_{\omega, b} \frac{\gamma||\omega||_2}{||\omega||_2}
\stopformula
\startformula
y_i(\omega^Tx_i + b) \ge \gamma||\omega||_2
\stopformula

Note that $\gamma_i||\omega||_2 = ||\omega^Tx_i + b||_2$ is proportional to $\omega$ and $b$, and $\gamma_i||\omega||_2$ is the only constraint parameter in the inequality, we can set  $\gamma_i||\omega||_2 = 1$ to obtain:
\startformula
\max_{\omega} \frac{1}{||\omega||_2}
\stopformula
\startformula
y_i(\omega^Tx_i + b) \ge 1
\stopformula
i.e.
\startformula
\min_{\omega} \frac{1}{2}||\omega||_2
\stopformula
\startformula
y_i(\omega^Tx_i + b) - 1\ge 0
\stopformula
Which is a standard convex quadratic programming problem.
If it has solution, the margin distance is $\frac{2}{||\omega||_2}$ and the data points of the minimum distance to the hyperplane are called \bold{support vectors}.

\subject{Lagrange Dual Function}
Consider the following nonlinear minimization or maximization problem:

Optimize
\startformula
\min_xf(x)
\stopformula
subject to
\startformula
{\displaystyle g_{i}(x^{*})\leq 0,{\text{ for }}i=1,\ldots ,m}
\stopformula
\startformula
{\displaystyle h_{j}(x^{*})=0,{\text{ for }}j=1,\ldots ,\ell \,\!}
\stopformula
The Lagrange function is defined as follows
\startformula
g: \bold{R}^m \times \bold{R}^\ell \rightarrow \bold{R}
\stopformula
\startformula
g(\mu, \lambda) = \inf_{x \in D} L(x, \mu, \lambda)
\stopformula
\startformula
= \inf_{x \in D} (f(x) + \sum_{i=1}^m\mu_ig_i(x) + \sum_{j=1}^{\ell}\lambda_jh_j(x))
\stopformula
$g$ is concave, can be $-\infty$ for some $\mu$, $\lambda$


\bold{Lower bound property} If $\mu_i \geq 0$, then $g(\mu, \lambda) \leq p^*$, $p^*$ is the optimal value of the primal problem. Denote the optimal value of dual problem is $d^*$, we have
\startformula
d^* = \max_{\mu, \lambda}\min_xL(x, \mu,\lambda) \leq \min_x\max_{\mu, \lambda}L(x, \mu,\lambda) = p^*
\stopformula


\bold{Strong duality} $d^* = p^*$ usually holds for convex problems. conditions that guarantee strong duality in convex problems are called \bold{constraint qualifications}.

\subject{Karush-Kuhn-Tucker Conditions}
\bold{Necessary conditions} Suppose $f$ and $g_i$ are continuously differentiable at a point $x^*$. If $x^*$ is a local optimum and the optimization problem satisfies some regularity conditions below, then there exist constants $\mu_i$ and $\lambda_j$, called KKT multipliers, such that
\startformula
{\displaystyle g_{i}(x^{*})\leq 0,{\text{ for }}i=1,\ldots ,m}
\stopformula
\startformula
{\displaystyle h_{j}(x^{*})=0,{\text{ for }}j=1,\ldots ,\ell \,\!}
\stopformula
\startformula
{\displaystyle -\nabla f(x^{*})=\sum _{i=1}^{m}\mu _{i}\nabla g_{i}(x^{*})+\sum _{j=1}^{\ell }\lambda _{j}\nabla h_{j}(x^{*}),}
\stopformula
\startformula
{\displaystyle \mu _{i}\geq 0,{\text{ for }}i=1,\ldots ,m}
\stopformula
\startformula
{\displaystyle \mu _{i}g_{i}(x^{*})=0,{\text{ for }}\;i=1,\ldots ,m.}
\stopformula
\bold{Sufficient conditions} In linear optimization problems, the necessary conditions are also sufficient for optimality.
\subject{Dual Problem}

By introducing the KKT multipliers $a_i \ge 0, i = 1,...,N$ for every inequality, define the Lagrange function:
\startformula
L(\omega, b, \alpha) = \frac{1}{2}||\omega||_2^2 - (\sum^N_{i=1}\alpha_iy_i(\omega^Tx + b) - \sum^N_{i=1}\alpha_i)
\stopformula
The dual of the primal problem is
\startformula
\max_\alpha\min_{\omega, b}L(\omega, b, \alpha)
\stopformula
Calculate the partial derivatives and assign it to 0
\startformula
\nabla_\omega L(\omega, b, \alpha) = \omega - \sum_{i=1}^N \alpha_i y_i x_i = 0
\stopformula
\startformula
\nabla_b L(\omega, b, \alpha) = \sum_{i=1}^N \alpha_i y_i = 0
\stopformula
By substituting $\omega$, we obtain the dual problem
\startformula
\min_\alpha \frac{1}{2}\sum_{i=1}^N\sum_{j=1}^N\alpha_i\alpha_jy_iy_j(x_i^Tx_j) - \sum_{i=1}^N \alpha_i
\stopformula
\startformula
\sum_{i=1}^N\alpha_iy_i = 0
\stopformula
\startformula
\alpha_i \geq 0, \text{    for    } i = 1,\ldots ,N
\stopformula
According to KKT condition, in the SVM primal problem, $f(x)$ and $g(x)$ is convex, $h(x)$ is affine. Thus KKT can be satisfied, and $d^* = p^*$. The optimal of dual problem is also the optimal of primal problem.

Denote $f^*(x) = \omega^* + b^*$ is the optimal SVM solution. If the KKT is met (solution is obtained). Then, for $\forall x_i \in T$ 
\startformula
\alpha_i \eq 0 \Leftrightarrow y_if^*(x_i) \geq 1
\stopformula
\startformula
0 \leq \alpha_i \leq C \Leftrightarrow y_if^*(x_i) \eq 1
\stopformula
\startformula
\alpha_i \eq C \Leftrightarrow y_if^*(x_i) \leq 1
\stopformula

\subject{Kernel Method}
Sometimes the data point distribution is not linear in input space. Sometimes by proper modeling, it could be linear after applying some mapping to another feature space.
\startformula
\phi(x) = \chi \rightarrow H
\stopformula
Apply the mapping to the dual form
\startformula
\min_\alpha \frac{1}{2}\sum_{i=1}^N\sum_{j=1}^N\alpha_i\alpha_jy_iy_j(\phi(x_i)^T\phi(x_j)) - \sum_{i=1}^N \alpha_i
\stopformula
\startformula
\sum_{i=1}^N\alpha_iy_i = 0
\stopformula
\startformula
\alpha_i \geq 0, \text{    for    } i = 1,\ldots ,N
\stopformula
Define kernel function
\startformula
K(x, z) = \phi(x)^T\phi(z)
\stopformula
Then mapping function is substituted
\startformula
\min_\alpha \frac{1}{2}\sum_{i=1}^N\sum_{j=1}^N\alpha_i\alpha_jy_iy_jK(x_i, x_j) - \sum_{i=1}^N \alpha_i
\stopformula
\startformula
\sum_{i=1}^N\alpha_iy_i = 0
\stopformula
\startformula
\alpha_i \geq 0, \text{    for    } i = 1,\ldots ,N
\stopformula
The necessary and sufficient conditions of that a symmetry mapping $K: \chi \times \chi \rightarrow \bold{R}$ can be a kernel function is that for $\forall x_i \in \chi, i = 1,\ldots, m$, the Gram Matrix of $K$
\startformula
K = [K(x_i, x_j)]_{m\times m}
\stopformula
is positive semidefinite.
In practice, we use Mercer's Theorem to obtain a more restrictive Mercer Kernel.

Or, we can design new kernels from existing kernels, such that $K_1: \chi \times \chi \rightarrow \bold{R}$ and $K_2: \chi \times \chi \rightarrow \bold{R}$ are kernels, functions below are also kernels
\startformula
K(x, z) = \alpha_1K_1(x, z)K_2(x, z) + \alpha_2K_1(x, z) + \alpha_3K_2(x, z) + \alpha_4f(x)f(z) + \alpha_5
\stopformula
\startformula
f: \chi \rightarrow \bold{R}
\stopformula
\startformula
\alpha_i \in \bold{R}^+
\stopformula
Or,
\startformula
K(x, z) = (K_1, (x, z) + \theta_1)^{\theta_2}, \theta_1 \in \bold{R}^+, \theta_2 \in \bold{N}
\stopformula
Or,
\startformula
K(x, z) = \exp(\frac{K_1(x, z)}{\sigma^2}), \sigma \in \bold{R}^+
\stopformula
Or,
\startformula
K(x, z) = \exp(-\frac{K_1(x, x) - 2K_1(x, z) + K_1(z, z)}{2\sigma^2}), \sigma \in \bold{R}^+
\stopformula
Or,
\startformula
K(x, z) = \frac{K_1(x, z)}{\sqrt{K_1(x, x)K_1(z, z)}}
\stopformula

\subject{Soft Margin}
We often encounter such data points that can be simply evaluated with linear models but the data set itself is not linearly separable, e.g. noise points. By adding a slack variable $\xi$ and penalty term $C$, original primal problem is modified to
\startformula
\min_{\omega, b,\xi} \frac{1}{2}||\omega||^2_2+ C\sum^N_{i=1}\xi_i
\stopformula
\startformula
\text{s.t.       } y_i(\omega^Tx_i + b) \geq 1 - \xi_i, i = 1,\ldots,N
\stopformula
\startformula
\xi_i \geq 0
\stopformula
It's noteworthy that some algorithms use quadratic form $\xi^2$ as the slack variable.


As it is still a convex minimization problem, by apply the KKT condition, the dual form can be obtained
\startformula
\min_\alpha \frac{1}{2}\sum_{i=1}^N\sum_{j=1}^N\alpha_i\alpha_jy_iy_j(x_i^Tx_j) - \sum_{i=1}^N \alpha_i
\stopformula
\startformula
\sum_{i=1}^N\alpha_iy_i = 0
\stopformula
\startformula
C \geq \alpha_i \geq 0, \text{    for    } i = 1,\ldots ,N
\stopformula
Compared to the hard margin dual problem:
\startformula
\min_\alpha \frac{1}{2}\sum_{i=1}^N\sum_{j=1}^N\alpha_i\alpha_jy_iy_j(x_i^Tx_j) - \sum_{i=1}^N \alpha_i
\stopformula
\startformula
\sum_{i=1}^N\alpha_iy_i = 0
\stopformula
\startformula
\alpha_i \geq 0, \text{    for    } i = 1,\ldots ,N
\stopformula
The two problem has the same representation of $\omega^*$ and $b^*$ to the primal problem solution
\startformula
\omega^* = \sum^N_{i = 1} \alpha^*_iy_ix_i
\stopformula
\startformula
b^* = y_i - \sum^N_{i=1}\alpha^*_iy_i(x_i^Tx_j)
\stopformula

The dual problem is rid of the slack variable $\xi$, and we can derive the hard margin form by setting the penalty term $C \rightarrow +\infty$ in the target equation of the soft margin form.


But most interestingly, by adding a slack variable to the primal problem we make the problem more "soft" -- however, in the dual problem -- the penalty $C$ is taken as the upper bound of $a_i$, which means, the higher the penalty, the larger the feasible range of $a_i$.

One might raise a question: Why should we add that $C$ in the soft margin dual problem?
The answer is, if the data set is not linearly separable, the unconstraint $\alpha$ will make negative infinity of target function $\min f(x) \rightarrow -\infty$.

\subject{Hinge loss function perspective}
The target function can be introduced by hinge loss function
\startformula
J(\theta) = [1 - \theta]_+
\stopformula
by assigning
\startformula
\theta_i = y_i(\omega^Tx_i + b)
\stopformula
and appending a weight decay (the squared $L^2$ norm of $||\omega||_2$)
\startformula
J(\omega, b) = \sum^N_{i=1}[1-y_i(\omega^Tx_i + b)]_+ + \lambda ||\omega||_2
\stopformula
Which is same as the primal problem of soft margin SVM. ($\xi_i = [1-y_i(\omega^Tx_i + b)]_+$).


Let $\lambda \rightarrow 0$, obtain the equality of the generalized hard margin form.


\bold{About weight decay} As mentioning about the weight decay -- if we are working on a polynomial regression (free dimension is sufficient) with no weight decay. The \bold{Moore-Penrose pseudo-inverse} will get the a solution with minimal $||\omega||_2$, and minimum $||Ax - y||_2$ if no perfect match can be found (hypothesis space is small).


\stoptext